\documentclass[11pt]{article}
\title{\LARGE \bf Solving Quadratic Equations}
\author{\Large  Ralph Howard}


\begin{document}
\maketitle 

It is well known how to solve polynomial equations of the first
degree. For a first degree equation ax+b=0 with a not equal to 0 the
solution is x=-b/a.  We now look at solving ax\^{}2+bx+c=0.


\textbf{To Prove:}
The equation ax\^{}2+bx+c=0 with a not equal to 0 as the solutions

x=(-b -+ sqrt(b\^{}2-4ac))/(2a)


\textbf{Proof:}
We use the method of completing the square to rewrite $ax^2+bx+c$.

ax\^{}2+bx+c = a ( x\^{}2 + b/a x)+c 

=a(x\^{}2 + b/a x + b/(2a) )\^{}2 -(b/(2a))\^{}2 )+c

=a(x + b/(2a))\^{}2 -  a(b/(2a)\^{}2+c

= a(x+b/(2a))\^{}2- (b\^{}2-4ac)/{4a}.

Therefore ax\^{}2+bx+c=0 can be rewritten as 

a(x+b/(2a))\^{}2 - (b\^{}2-4ac)/(4a)=0, 

which can in turn  be rearranged as

(x+b/(2a))\^{}2= (b\^{}2-4ac)/(4a\^{}2).

Taking square roots gives

x+(b)/(2a)= (+- sqrt(b\^{}2-4ac))/(2a)

which implies

x=(-b+- sqrt(b\^{}2-4ac))(2a)

as required.





\end{document}
