\documentclass[11pt]{article}
\title{\LARGE \bf Solving Quadratic Equations}
\author{\Large  Ralph Howard}
\usepackage{amsmath}

\begin{document}
\maketitle 

It is well known how to solve polynomial equations of the first
degree. For a first degree equation $ax+b=0$ with $a \neq 0$ the
solution is $x=-\frac{b}{a}$.  We now look at solving $ax^2+bx+c=0$.

\textbf{To Prove:}
The equation $ax^2+bx+c=0$ with $a \neq 0$ has the solutions

\begin{equation}
x=\frac{-b \pm \sqrt{b^2-4ac}}{2a}
\end{equation}

\textbf{Proof:}
We use the method of completing the square to rewrite $ax^2+bx+c$.

\begin{align}
ax^2+bx+c &= a \left( x^2 + \frac{b}{a} x \right)+c \\
&=a\left(x^2 + \frac{b}{a} x + \frac{b}{2a} \right)^2 -\frac{b}{2a}^2 +c\\
&=a\left(x + \frac{b}{2a}\right)^2 -  a\left(\frac{b}{2a}\right)^2+c \\
&=a\left(x+\frac{b}{2a}\right)^2 - \frac{\left(b^2-4ac\right)}{4a}.
\end{align}

Therefore $ax^2+bx+c=0$ can be rewritten as 
\begin{equation}
a\left(x+\frac{b}{2a}\right)^2 - \frac{\left(b^2-4ac\right)}{4a}=0,
\end{equation}
which can in turn  be rearranged as
\begin{equation}
\left(x+\frac{b}{2a}\right)^2= \frac{\left(b^2-4ac\right)}{4a^2}.
\end{equation}
Taking square roots gives
\begin{equation}
x+\frac{b}{2a}= \pm \frac{\sqrt{b^2-4ac}}{2a}
\end{equation}
which implies
\begin{equation}
x=\frac{-b \pm \sqrt{b^2-4ac}}{2a}
\end{equation}
as required.

\end{document}
